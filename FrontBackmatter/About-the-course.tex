%*******************************************************
% About this course
%*******************************************************
\pdfbookmark[1]{About this course}{About this course}
\begingroup
\let\clearpage\relax
\let\cleardoublepage\relax
\let\cleardoublepage\relax

\chapter*{About the course}
This course was developed in the \href{http://www.eng.ed.ac.uk}{School of Engineering} at \href{http://www.ed.ac.uk}{The University of Edinburgh} to provide appropriate material for teaching \mlab \footnote{\mlab\textregistered\ is a registered trademark of MathWorks\\} in all engineering disciplines as well as to a wider audience. It is a self-study, self-paced course that emphasises responsible learning. The course material consists of this document used in conjunction with a series of video screencasts that are hosted on the School of Engineering YouTube channel.\\

\textsc{Who should use this document?}
\begin{quote}
This document is targeted at those with no prior knowledge of \mlab, and no previous programming experience. The aim, upon completion of the course, is to be competent using the most common features in \mlab and be able to apply them to solve engineering problems.
\end{quote}

\textsc{What is in this document?}
\begin{quote}
This document forms part of a self-study course to help you get started with \mlab. It should be used along with the supporting \href{https://www.youtube.com/playlist?list=PLDlE-GBjzmBZsxFKZfp6Y59qDOJVIh4RN}{video screencasts} and any additional material that maybe hosted on your institutions virtual learning environment. The main body of this document contains the fundamental topics for the course and there are also several more advanced topics given in appendices.
\end{quote}

\textsc{How to use this document?}
\begin{quote}
This document contains different elements designed to make your learning experience as smooth as possible. To benefit the most from these elements you are encouraged to use the online PDF version of this document \graffito{You can use the commenting tools in Adobe Reader to add your own notes to this PDF document}. One of the first things you'll notice is that this document contains many links, those in \textcolor{webbrown}{red} indicate a link to online material, and those in \textcolor{RoyalBlue}{blue} indicate a link to another section of this document. 

A key part of this course are the screencasts, \href{http://en.wikipedia.org/wiki/Screencast}{which are video screen captures} (http://en.wikipedia.org/wiki/Screencast). In this document screencasts are indicated by a link in a blue box with a clapperboard icon, like the example shown. 
\end{quote}
\graffito{Watching the screencasts and trying the examples for yourself will help you develop your skills in \mlab more quickly!}
\addtolength{\parindent}{-4mm}
\fcolorbox{myborderblue}{myblue}{%
\begin{minipage}{\linewidth}
\begin{minipage}{6mm}
\includegraphics[scale=0.03]{Graphics/General/screencast_icon}
\end{minipage}
\href{https://youtu.be/vtTlvq6s7a4}{\textit{Getting started}}\\
(https://youtu.be/vtTlvq6s7a4)
%\href{http://www.eng.ed.ac.uk/teaching/courses/matlab/getting-started.shtml}{\textit{Getting started}}\\
%(http://www.eng.ed.ac.uk/teaching/courses/matlab/getting-started.shtml)
\end{minipage}%
}\\
\addtolength{\parindent}{4mm}

\begin{quote}
Clicking on a link to a screencast will take you to the appropriate page on the course website where you will see the opening image to a University of Edinburgh screencast presented in the video player (Figure~\ref{fig:screencast}). Watching and learning from the screencasts are an essential part of the course and will help you develop your skills in \mlab more quickly.
\begin{figure}
	\myfloatalign
	\includegraphics[width=\linewidth]{Graphics/General/screencast}
	\caption{A University of Edinburgh screencast}
	\label{fig:screencast}
\end{figure}

You will also notice two other types of blue box environments in this document: one is for \textit{Hints and Tips} (with a question mark icon), and the other contains exercises that you should complete (with an inkwell icon).
\end{quote}
\addtolength{\parindent}{-4mm}
\fcolorbox{myborderblue}{myblue}{%
\begin{minipage}{\linewidth}
\begin{minipage}{6mm}
\includegraphics[scale=0.03]{Graphics/General/help_icon}
\end{minipage}
\textit{Hints and Tips} \\
Throughout this document you will also see \textit{Hints and Tips} boxes like this one. Please read these as they contain \textit{useful} hints!
\end{minipage}%
}\\
\addtolength{\parindent}{4mm}\\

\addtolength{\parindent}{-4mm}
\fcolorbox{myborderblue}{myblue}{%
\begin{minipage}{\linewidth}
\begin{minipage}{6mm}
\includegraphics[scale=0.035]{Graphics/General/exercise_icon}
\end{minipage}
\textit{An example exercise}

\begin{minipage}{6mm}
\includegraphics[scale=0.03]{Graphics/General/screencast_icon}
\end{minipage}
\textit{\textcolor{webbrown}{Example exercise solutions}}
\end{minipage}%
}\\
\addtolength{\parindent}{4mm}

\begin{quote}
Additionally there are grey box environments in this document. Like the example shown (Listing~\ref{lst:lst_example}), these contain code listings that demonstrate actual \mlab code. Line numbers are given to the left of the listings to make is simpler to refer to specific bits of code. Very often you will be required to copy and paste the listing into \mlab and try running it for yourself.
\end{quote}
\begin{lstlisting}[caption={Example of a code listing},label=lst:lst_example]	 
>> 5+5
ans =
	 10
\end{lstlisting}
\vspace{5mm}

\textsc{Sources of help and further reading}
\begin{quote}
There are a huge number of textbooks published on the subject of \mlab! A user-friendly textbook that provides a good introduction to \mlab is:\graffito{Available from \href{http://www.amazon.co.uk/MATLAB-Introduction-Applications-Amos-Gilat/dp/0470108770/ref=sr_1_2?ie=UTF8&qid=1322656950&sr=8-2}{Amazon} for c.\pounds15}
\begin{itemize}
\item Gilat, A. (2008). \textit{\mlab: An Introduction With Applications}. John Wiley \& Sons, Inc., 3rd edition. 
\end{itemize}
There are a couple of further textbooks listed in the \hyperref[app:bibliography]{Bibliography} section at the end of this document. However, throughout this course and beyond, the most important source of help is the documentation built-in to \mlab. It is easily searchable, and because \mlab contains many built-in functions it is worth checking out before starting to write your own code.
\begin{itemize}
\item \mlab help documentation\\(\href{http://www.mathworks.com/access/helpdesk/help/techdoc/}{http://www.mathworks.com/access/helpdesk/help/techdoc/})\\ Accessed through the help menu in \mlab, or online.
\item \mlab \textsc{Central}\\(\href{http://www.mathworks.co.uk/matlabcentral/}{http://www.mathworks.co.uk/matlabcentral/})\\An open exchange for users, with code snippets, help forums and blogs. A great place to search for specific help!
\end{itemize}
\end{quote}

% List of books
\nocite{Pratap:2006fk} 
\nocite{Magrab:2005fk}
\nocite{Moore:2009fk}

\newpage
\textsc{Development of the course}
\begin{quote}
The development of this course was funded through The Edinburgh Fund Small Project Grant which is part of The University of Edinburgh Campaign \\ (\href{http://www.edinburghcampaign.com/alumni-giving/grants}{http://www.edinburghcampaign.com/alumni-giving/grants}).

The material for this course was developed by \href{http://www.bath.ac.uk/chem-eng/people/duren/}{Prof. Tina D\"uren}, \href{https://www.eng.ed.ac.uk/about/people/dr-antonis-giannopoulos}{Dr. Antonis Giannopoulos}, \href{Prof. Guillermo Rein}{https://www.imperial.ac.uk/people/g.rein}, \href{https://www.eng.ed.ac.uk/about/people/prof-john-thompson}{Prof. John Thompson}, and \href{https://www.northumbria.ac.uk/about-us/our-staff/w/craig-warren/}{Dr. Craig Warren}. The original screencasts were created by Dr. Craig Warren. Revised screencasts (to reflect an update to the \mlab User Interface - \mlab R2013a) were made by Joe Burchell.
\end{quote}

\textsc{License and usage of this course}
\begin{quote}
This work is licensed under the Creative Commons Attribution-NonCommercial-ShareAlike 3.0 Unported License. To view a copy of this license, visit \href{http://creativecommons.org/licenses/by-nc-sa/3.0/ <http://creativecommons.org/licenses/by-nc-sa/3.0/}{http://creativecommons.org/licenses/by-nc-sa/3.0/} or send a letter to Creative Commons, PO Box 1866, Mountain View, CA 94042, USA.

Under the terms of the license we would be grateful if you would cite the following paper:
\begin{itemize}
\item Warren, C. (2014). \mlab for Engineers: Development of an Online, Interactive, Self-study Course. Engineering Education, 9(1), 86-93. 
\end{itemize}
\end{quote}

\textsc{Acknowledgments}
\begin{quote}
An Interactive Introduction to \mlab makes use of the \href{http://www.miede.de/#classicthesis}{classicthesis template} for LaTeX, which was created by Prof. Dr.-Ing. Andr\'{e} Miede.
\end{quote}

\endgroup